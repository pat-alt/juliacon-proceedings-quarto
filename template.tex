% Options for packages loaded elsewhere
\PassOptionsToPackage{unicode}{hyperref}
\PassOptionsToPackage{hyphens}{url}
\PassOptionsToPackage{dvipsnames,svgnames,x11names}{xcolor}
%

\documentclass{juliacon}

\usepackage{amsmath,amssymb}
\usepackage{setspace}
\usepackage{iftex}
\ifPDFTeX
  \usepackage[T1]{fontenc}
  \usepackage[utf8]{inputenc}
  \usepackage{textcomp} % provide euro and other symbols
\else % if luatex or xetex
  \usepackage{unicode-math}
  \defaultfontfeatures{Scale=MatchLowercase}
  \defaultfontfeatures[\rmfamily]{Ligatures=TeX,Scale=1}
\fi
\usepackage{lmodern}
\ifPDFTeX\else  
    % xetex/luatex font selection
\fi
% Use upquote if available, for straight quotes in verbatim environments
\IfFileExists{upquote.sty}{\usepackage{upquote}}{}
\IfFileExists{microtype.sty}{% use microtype if available
  \usepackage[]{microtype}
  \UseMicrotypeSet[protrusion]{basicmath} % disable protrusion for tt fonts
}{}
\makeatletter
\@ifundefined{KOMAClassName}{% if non-KOMA class
  \IfFileExists{parskip.sty}{%
    \usepackage{parskip}
  }{% else
    \setlength{\parindent}{0pt}
    \setlength{\parskip}{6pt plus 2pt minus 1pt}}
}{% if KOMA class
  \KOMAoptions{parskip=half}}
\makeatother
\usepackage{xcolor}
\setlength{\emergencystretch}{3em} % prevent overfull lines
\setcounter{secnumdepth}{5}
% Make \paragraph and \subparagraph free-standing
\ifx\paragraph\undefined\else
  \let\oldparagraph\paragraph
  \renewcommand{\paragraph}[1]{\oldparagraph{#1}\mbox{}}
\fi
\ifx\subparagraph\undefined\else
  \let\oldsubparagraph\subparagraph
  \renewcommand{\subparagraph}[1]{\oldsubparagraph{#1}\mbox{}}
\fi


\providecommand{\tightlist}{%
  \setlength{\itemsep}{0pt}\setlength{\parskip}{0pt}}\usepackage{longtable,booktabs,array}
\usepackage{calc} % for calculating minipage widths
% Correct order of tables after \paragraph or \subparagraph
\usepackage{etoolbox}
\makeatletter
\patchcmd\longtable{\par}{\if@noskipsec\mbox{}\fi\par}{}{}
\makeatother
% Allow footnotes in longtable head/foot
\IfFileExists{footnotehyper.sty}{\usepackage{footnotehyper}}{\usepackage{footnote}}
\makesavenoteenv{longtable}
\usepackage{graphicx}
\makeatletter
\def\maxwidth{\ifdim\Gin@nat@width>\linewidth\linewidth\else\Gin@nat@width\fi}
\def\maxheight{\ifdim\Gin@nat@height>\textheight\textheight\else\Gin@nat@height\fi}
\makeatother
% Scale images if necessary, so that they will not overflow the page
% margins by default, and it is still possible to overwrite the defaults
% using explicit options in \includegraphics[width, height, ...]{}
\setkeys{Gin}{width=\maxwidth,height=\maxheight,keepaspectratio}
% Set default figure placement to htbp
\makeatletter
\def\fps@figure{htbp}
\makeatother

\usepackage{orcidlink}
\definecolor{mypink}{RGB}{219, 48, 122}
\makeatletter
\@ifpackageloaded{caption}{}{\usepackage{caption}}
\AtBeginDocument{%
\ifdefined\contentsname
  \renewcommand*\contentsname{Table of contents}
\else
  \newcommand\contentsname{Table of contents}
\fi
\ifdefined\listfigurename
  \renewcommand*\listfigurename{List of Figures}
\else
  \newcommand\listfigurename{List of Figures}
\fi
\ifdefined\listtablename
  \renewcommand*\listtablename{List of Tables}
\else
  \newcommand\listtablename{List of Tables}
\fi
\ifdefined\figurename
  \renewcommand*\figurename{Figure}
\else
  \newcommand\figurename{Figure}
\fi
\ifdefined\tablename
  \renewcommand*\tablename{Table}
\else
  \newcommand\tablename{Table}
\fi
}
\@ifpackageloaded{float}{}{\usepackage{float}}
\floatstyle{ruled}
\@ifundefined{c@chapter}{\newfloat{codelisting}{h}{lop}}{\newfloat{codelisting}{h}{lop}[chapter]}
\floatname{codelisting}{Listing}
\newcommand*\listoflistings{\listof{codelisting}{List of Listings}}
\makeatother
\makeatletter
\makeatother
\makeatletter
\@ifpackageloaded{caption}{}{\usepackage{caption}}
\@ifpackageloaded{subcaption}{}{\usepackage{subcaption}}
\makeatother
\ifLuaTeX
  \usepackage{selnolig}  % disable illegal ligatures
\fi
\IfFileExists{bookmark.sty}{\usepackage{bookmark}}{\usepackage{hyperref}}
\IfFileExists{xurl.sty}{\usepackage{xurl}}{} % add URL line breaks if available
\urlstyle{same} % disable monospaced font for URLs
\hypersetup{
  pdftitle={Juliacon Proceedings in Quarto},
  pdfauthor={Patrick Altmeyer},
  pdfkeywords={Template, Demo, Quarto, JuliaCon},
  colorlinks=true,
  linkcolor={blue},
  filecolor={Maroon},
  citecolor={Blue},
  urlcolor={Blue},
  pdfcreator={LaTeX via pandoc}}


\title{Juliacon Proceedings in Quarto}

\author[1]{Patrick Altmeyer}
\affil[1]{Delft University of Technology}
\date{}
\begin{document}
\maketitle

% Abstract
\begin{abstract}

This document is a template demonstrating the
\texttt{juliacon-proceedings} format. It is a standard Quarto document
composed of Markdown, LaTeX and code chunks. It is rendered to PDF and
HTML. The PDF is rendered using the \texttt{juliacon-proceedings-pdf}
format, which is based on the \texttt{juliacon-proceedings} LaTeX class.
The HTML is rendered using the \texttt{juliacon-proceedings-html}
format, which is based on the \texttt{juliacon-proceedings} HTML
template.
\end{abstract}

% Keywords
\JCONkeywords{Template, Demo, Quarto, JuliaCon}

% Hypersetup
\hypersetup{
    pdftitle = {Juliacon Proceedings in Quarto},
    pdfsubject = {JuliaCon \@year Proceedings},
    pdfauthor = {Patrick Altmeyer},
    pdfkeywords = {Template, Demo, Quarto, JuliaCon},
}

\setcounter{page}{1}

\setstretch{1}
\section{Introduction}\label{sec-intro}

This is a template for writing a JuliaCon Proceedings article in Quarto.
This is a proof-of-concept for how we could use Quarto for JuliaCon
proceedings. For current submissions, please ignore this repo and follow
the official instructions
\href{https://github.com/JuliaCon/JuliaConSubmission.jl}{here}.

\subsection{What is Quarto?}\label{what-is-quarto}

\href{https://quarto.org/}{Quarto} makes it easy to write reproducible
documents that can be rendered to PDF, HTML, Word and more. It is based
on Markdown, which is easy to learn and write. It also supports LaTeX,
which is useful for more advanced formatting. As this extension
demonstrates, Quarto is also very flexible and can be extended with
custom templates and styles.

\subsection{Why Quarto?}\label{why-quarto}

By embracing Quarto, JuliaCon Proceedings can set an example for how to
write reproducible documents. We would not only make it easier for
authors to write their submissions but also open the door for more
advanced features such as interactive figures and executable code blocks
in HTML documents.

\subsection{About this template}\label{about-this-template}

This template is based on the existing
\href{https://github.com/JuliaCon/JuliaConSubmission.jl}{JuliaCon
Proceedings LaTeX template}.

\subsection{How to use this template}\label{how-to-use-this-template}

\begin{quote}
The remainder of this document repeats relevant elements from the
{[}JuliaCon Proceedings LaTeX template{]}.
\end{quote}

\section{The JuliaCon Article Class}\label{sec-documentclass}

The \texttt{juliacon} class file preserves the standard LATEX\{\}
interface such that any document that can be produced using the standard
LATEX\{\} article class can also be produced with the class file.

It is likely that the make up will change after file submission. For
this reason, we ask you to ignore details such as slightly long lines,
page stretching, or figures falling out of synchronization, as these
details can be dealt with at a later stage.

Use should be made of symbolic references (\verb|\ref| ) in order to
protect against late changes of order, etc.

\section{Executable Code}\label{executable-code}

Ideally, we would be able to tab into Quarto's existing support for
executable, cross-referenceable code chunks.

\begin{quote}
To create cross-referenceable code listings from executable code blocks,
use \texttt{lst-label} and \texttt{lst-cap}. ---
\href{https://quarto.org/docs/prerelease/1.4/crossref.html\#cross-referenceable-listings-of-executable-code-blocks}{Quarto
Docs}
\end{quote}

For example, the following code block will be labeled as \texttt{lst-1}
and calling \texttt{@lst-1} will render as Listing~\ref{lst-1}. This
ensures that things stay up-to-date and potential errors in the code are
identified early: as you render your document, the code will be executed
and the output will be inserted into the document.

\begin{codelisting}

\caption{\label{lst-1}A listing caption}

\centering{

\begin{verbatim}
using Plots

x = -3.0:0.01:3.0
y = rand(length(x));
\end{verbatim}

}

\end{codelisting}%

Unfortunately, I don't know how to use the special environment that is
already defined for Julia in this context.

\section{Code Listings}\label{code-listings}

The special environment that is already defined for Julia code can still
be used as before.

\begin{verbatim}
\begin{lstlisting}[
    language = Julia, 
    numbers=left, 
    label={lst:exmplg}, 
    caption={Example Code Block.}
]
using Plots

x = -3.0:0.01:3.0
y = rand(length(x))
plot(x, y)
\end{lstlisting}
\end{verbatim}
\begin{lstlisting}[
    language = Julia, 
    numbers=left, 
    label={lst:exmplg}, 
    caption={Example Code Block.}
]
using Plots

x = -3.0:0.01:3.0
y = rand(length(x))
plot(x, y)
\end{lstlisting}



\end{document}
