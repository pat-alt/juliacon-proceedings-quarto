% Options for packages loaded elsewhere
\PassOptionsToPackage{unicode}{hyperref}
\PassOptionsToPackage{hyphens}{url}
\PassOptionsToPackage{dvipsnames,svgnames,x11names}{xcolor}
%

\documentclass{juliacon}

\usepackage{amsmath,amssymb}
\usepackage{iftex}
\ifPDFTeX
  \usepackage[T1]{fontenc}
  \usepackage[utf8]{inputenc}
  \usepackage{textcomp} % provide euro and other symbols
\else % if luatex or xetex
  \usepackage{unicode-math}
  \defaultfontfeatures{Scale=MatchLowercase}
  \defaultfontfeatures[\rmfamily]{Ligatures=TeX,Scale=1}
\fi
\usepackage{lmodern}
\ifPDFTeX\else  
    % xetex/luatex font selection
\fi
% Use upquote if available, for straight quotes in verbatim environments
\IfFileExists{upquote.sty}{\usepackage{upquote}}{}
\IfFileExists{microtype.sty}{% use microtype if available
  \usepackage[]{microtype}
  \UseMicrotypeSet[protrusion]{basicmath} % disable protrusion for tt fonts
}{}
\makeatletter
\@ifundefined{KOMAClassName}{% if non-KOMA class
  \IfFileExists{parskip.sty}{%
    \usepackage{parskip}
  }{% else
    \setlength{\parindent}{0pt}
    \setlength{\parskip}{6pt plus 2pt minus 1pt}}
}{% if KOMA class
  \KOMAoptions{parskip=half}}
\makeatother
\usepackage{xcolor}
\setlength{\emergencystretch}{3em} % prevent overfull lines
\setcounter{secnumdepth}{5}
% Make \paragraph and \subparagraph free-standing
\ifx\paragraph\undefined\else
  \let\oldparagraph\paragraph
  \renewcommand{\paragraph}[1]{\oldparagraph{#1}\mbox{}}
\fi
\ifx\subparagraph\undefined\else
  \let\oldsubparagraph\subparagraph
  \renewcommand{\subparagraph}[1]{\oldsubparagraph{#1}\mbox{}}
\fi


\providecommand{\tightlist}{%
  \setlength{\itemsep}{0pt}\setlength{\parskip}{0pt}}\usepackage{longtable,booktabs,array}
\usepackage{calc} % for calculating minipage widths
% Correct order of tables after \paragraph or \subparagraph
\usepackage{etoolbox}
\makeatletter
\patchcmd\longtable{\par}{\if@noskipsec\mbox{}\fi\par}{}{}
\makeatother
% Allow footnotes in longtable head/foot
\IfFileExists{footnotehyper.sty}{\usepackage{footnotehyper}}{\usepackage{footnote}}
\makesavenoteenv{longtable}
\usepackage{graphicx}
\makeatletter
\def\maxwidth{\ifdim\Gin@nat@width>\linewidth\linewidth\else\Gin@nat@width\fi}
\def\maxheight{\ifdim\Gin@nat@height>\textheight\textheight\else\Gin@nat@height\fi}
\makeatother
% Scale images if necessary, so that they will not overflow the page
% margins by default, and it is still possible to overwrite the defaults
% using explicit options in \includegraphics[width, height, ...]{}
\setkeys{Gin}{width=\maxwidth,height=\maxheight,keepaspectratio}
% Set default figure placement to htbp
\makeatletter
\def\fps@figure{htbp}
\makeatother

\usepackage{orcidlink}
\definecolor{mypink}{RGB}{219, 48, 122}
\makeatletter
\@ifpackageloaded{caption}{}{\usepackage{caption}}
\AtBeginDocument{%
\ifdefined\contentsname
  \renewcommand*\contentsname{Table of contents}
\else
  \newcommand\contentsname{Table of contents}
\fi
\ifdefined\listfigurename
  \renewcommand*\listfigurename{List of Figures}
\else
  \newcommand\listfigurename{List of Figures}
\fi
\ifdefined\listtablename
  \renewcommand*\listtablename{List of Tables}
\else
  \newcommand\listtablename{List of Tables}
\fi
\ifdefined\figurename
  \renewcommand*\figurename{Figure}
\else
  \newcommand\figurename{Figure}
\fi
\ifdefined\tablename
  \renewcommand*\tablename{Table}
\else
  \newcommand\tablename{Table}
\fi
}
\@ifpackageloaded{float}{}{\usepackage{float}}
\floatstyle{ruled}
\@ifundefined{c@chapter}{\newfloat{codelisting}{h}{lop}}{\newfloat{codelisting}{h}{lop}[chapter]}
\floatname{codelisting}{Listing}
\newcommand*\listoflistings{\listof{codelisting}{List of Listings}}
\makeatother
\makeatletter
\makeatother
\makeatletter
\@ifpackageloaded{caption}{}{\usepackage{caption}}
\@ifpackageloaded{subcaption}{}{\usepackage{subcaption}}
\makeatother
\ifLuaTeX
  \usepackage{selnolig}  % disable illegal ligatures
\fi
\IfFileExists{bookmark.sty}{\usepackage{bookmark}}{\usepackage{hyperref}}
\IfFileExists{xurl.sty}{\usepackage{xurl}}{} % add URL line breaks if available
\urlstyle{same} % disable monospaced font for URLs
\hypersetup{
  pdftitle={Juliacon Proceedings in Quarto},
  pdfauthor={Patrick Altmeyer; Cynthia C. S. Liem},
  pdfkeywords={Template, Demo, Quarto, JuliaCon},
  colorlinks=true,
  linkcolor={blue},
  filecolor={Maroon},
  citecolor={Blue},
  urlcolor={Blue},
  pdfcreator={LaTeX via pandoc}}


\title{Juliacon Proceedings in Quarto}

\author[1]{Patrick Altmeyer}
\author[1]{Cynthia C. S. Liem}
\affil[1]{Delft University of Technology}
\date{}
\begin{document}
\maketitle

% Abstract
\begin{abstract}
This document is a Quarto template demonstrating the
\texttt{juliacon-proceedings} format.
\end{abstract}

% Keywords
\JCONkeywords{Template, Demo, Quarto, JuliaCon}

% Hypersetup
\hypersetup{
    pdftitle = {Juliacon Proceedings in Quarto},
    pdfsubject = {JuliaCon 2022 Proceedings},
    pdfauthor = {Patrick Altmeyer, Cynthia C. S. Liem},
    pdfkeywords = {Template, Demo, Quarto, JuliaCon},
}

\setcounter{page}{1}

\section{Introduction}\label{sec-intro}

\emph{TODO} Create a template that demonstrates the appearance,
formatting, layout, and functionality of your format. Learn more about
journal formats at \url{https://quarto.org/docs/journals/}.

\subsection{Proof-of-Concept}\label{proof-of-concept}

This is a proof-of-concept for how we could use Quarto for JuliaCon
proceedings. For current submissions, please ignore this repo and follow
the official instructions
\href{https://github.com/JuliaCon/JuliaConSubmission.jl}{here}.

\subsection{Executable Code}\label{executable-code}

Ideally, we would be able to tab into Quarto's existing support for
executable, cross-referenceable code chunks.

\begin{quote}
To create cross-referenceable code listings from executable code blocks,
use \texttt{lst-label} and \texttt{lst-cap}. ---
\href{https://quarto.org/docs/prerelease/1.4/crossref.html\#cross-referenceable-listings-of-executable-code-blocks}{Quarto
Docs}
\end{quote}

For example, the following code block will be labeled as \texttt{lst-1}
and calling \texttt{@lst-1} will render as Listing~\ref{lst-1}. This
ensures that things stay up-to-date and potential errors in the code are
identified early: as you render your document, the code will be executed
and the output will be inserted into the document.

\begin{codelisting}

\caption{\label{lst-1}A listing caption}

\centering{

\begin{verbatim}
using Plots

x = -3.0:0.01:3.0
y = rand(length(x))
plot(x, y)
\end{verbatim}

}

\end{codelisting}%

Unfortunately, I don't know how to use the special environment that is
already defined for Julia in this context.

\subsection{Code Listings}\label{code-listings}

The special environment that is already defined for Julia code can still
be used as before.

\begin{verbatim}
\begin{lstlisting}[
    language = Julia, 
    numbers=left, 
    label={lst:exmplg}, 
    caption={Example Code Block.}
]
using Plots

x = -3.0:0.01:3.0
y = rand(length(x))
plot(x, y)
\end{lstlisting}
\end{verbatim}
\begin{lstlisting}[
    language = Julia, 
    numbers=left, 
    label={lst:exmplg}, 
    caption={Example Code Block.}
]
using Plots

x = -3.0:0.01:3.0
y = rand(length(x))
plot(x, y)
\end{lstlisting}



\end{document}
