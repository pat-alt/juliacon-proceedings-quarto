% JuliaCon proceedings template
\documentclass{juliacon}
\setcounter{page}{1}
% **************GENERATED FILE, DO NOT EDIT**************

\def\@journalName{Proceedings of JuliaCon}
\def\@volume{1}
\def\@issue{1}
\def\@year{2022}% **************GENERATED FILE, DO NOT EDIT**************

\title{My JuliaCon proceeding}

\author[1]{1st author}
\author[1, 2]{2nd author}
\author[2]{3rd author}
\affil[1]{University}
\affil[2]{National Lab}

\keywords{Julia, Optimization, Game theory, Compiler}

\hypersetup{
pdftitle = {My JuliaCon proceeding},
pdfsubject = {JuliaCon 2022 Proceedings},
pdfauthor = {1st author, 2nd author, 3rd author},
pdfkeywords = {Julia, Optimization, Game theory, Compiler},
}
\begin{document}
\maketitle
\section{Introduction}\label{sec-intro}

\emph{TODO} Create a template that demonstrates the appearance,
formatting, layout, and functionality of your format. Learn more about
journal formats at \url{https://quarto.org/docs/journals/}.

\subsection{Proof-of-Concept}\label{proof-of-concept}

This is a proof-of-concept for how we could use Quarto for JuliaCon
proceedings. For current submissions, please ignore this repo and follow
the official instructions
\href{https://github.com/JuliaCon/JuliaConSubmission.jl}{here}
\end{document}
